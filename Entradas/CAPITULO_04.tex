\chapter{Estudio experimental de Turbulencia de Ondas} % REVISADO 1 (falta espectro) Resultados 
\thispagestyle{plain} 


Las experiencias realizadas para esta Tesis se corresponden al estudio estadístico de las propiedades de 

\clearpage 

\section{Montaje para Turbulencia de Ondas} 

\subsection{Montaje}
El armado experimental consistió, como se muestra en la Figura \ref{fig:montajewaveturbulence-fondo}, en una cuba de acrílico de 20x20 cm$^2$ llena hasta una altura de $h_0=(4.45\pm0.05)$ cm de agua destilada de 


\begin{figure}[!ht]
	\centering
	\includegraphics[width=0.879506\linewidth]{"Figuras/Setups/Montaje_wave_turbulence _fondo"}
	\caption{}
	\label{fig:montajewaveturbulence-fondo}
\end{figure}







\subsection{Inyección de energía}  
Para la inyección de energía en el sistema 

% Se eligió inyectar energía en el sistema mediante un forzado a escalas grandes. Para este forzado

\begin{figure}[th!]
	\centering
	\includegraphics[width=0.87\linewidth]{Figuras/Forzado/curva_forzado}
	\caption{(a) Curva típica para el forzado del fluido. En los paneles inferiores se muestra su contenido en frecuencia, con su amplitud (b) limitada al rango de 0-6 Hz en este caso , y su fase (c) aleatoria.}
	\label{fig:curvaforzado}
\end{figure}


\subsection{Juegos de Datos} 



\section{Series Temporales} 
Una serie temporal típica de las medidas para este tipo de forzados se muestra en la Figura \ref{fig:pdfs-cuba-chica-v2}. 


\begin{figure}[th!]
	\centering
	\includegraphics[width=0.987\linewidth]{"Figuras/La serie temporal/La_serie_temporal"}
	\caption{}
	\label{fig:laserietemporal}
\end{figure}



\begin{figure}[!ht]
	\centering
	\includegraphics[width=0.9897\linewidth]{"Figuras/PDFS/PDFS cuba chica v2"}
	\caption{}
	\label{fig:pdfs-cuba-chica-v2}
\end{figure}



\section{Wave Steepness} 

\begin{figure}[!ht]
	\centering
	\includegraphics[width=0.9897\linewidth]{"Figuras/Wave Steepness/Work_flow"}
	\caption{}
	\label{fig:workflow}
\end{figure}



\begin{figure}[!ht]
	\centering
	\includegraphics[width=0.9897\linewidth]{"Figuras/Wave Steepness/PDFs"}
	\caption{}
	\label{fig:pdfs}
\end{figure}




\begin{figure}[!ht]
	\centering
	\includegraphics[width=0.9897\linewidth]{"Figuras/Wave Steepness/Joint_porogression"}
	\caption{}
	\label{fig:jointporogression}
\end{figure}



\begin{figure}[!ht]
	\centering
	\includegraphics[width=0.9897\linewidth]{"Figuras/Wave Steepness/Mean_wave_steepness"}
	\caption{}
	\label{fig:meanwavesteepness}
\end{figure}




\section{Cascadas de energía}




\begin{figure}[!ht]
	\centering
	\includegraphics[width=0.9897\linewidth]{"Figuras/Espectros/PSDs/Cuba chica v2"}
	\caption{}
	\label{fig:cuba-chica-v2}
\end{figure}


\begin{figure}[!ht]
	\centering
	\includegraphics[width=0.9897\linewidth]{Figuras/Espectros/Workflow/Workflow_v2}
	\caption{}
	\label{fig:workflowv2}
\end{figure} 



\begin{figure}[!ht]
	\centering
	\includegraphics[width=0.9897\linewidth]{Figuras/Espectros/Cutoff/Cutoff}
	\caption{}
	\label{fig:cutoff}
\end{figure}



\begin{figure}[!ht]
	\centering
	\includegraphics[width=0.9897\linewidth]{Figuras/Espectros/Exponentes/Exponentes_apilados}
	\caption{}
	\label{fig:exponentesapilados}
\end{figure}



\begin{figure}[!ht]
	\centering
	\includegraphics[width=0.9897\linewidth]{"Figuras/Espectros/Frecuencia de cruce/Frecuencias de Cruce"}
	\caption{}
	\label{fig:frecuencias-de-cruce}
\end{figure}


\section{Interacción a 3 y 4 ondas} 

% \missingfigure{CUBOS} 
\begin{figure}[th!]
	\centering
	\includegraphics[width=0.987\linewidth]{Figuras/Correlaciones/Cubos}
	\caption{}
	\label{fig:cubos}
\end{figure}




\begin{figure}[!ht]
	\centering
	\includegraphics[width=0.9897\linewidth]{Figuras/Correlaciones/Triadas}
	\caption{}
	\label{fig:triadas}
\end{figure}

\begin{figure}[!ht]
	\centering
	\includegraphics[width=0.9897\linewidth]{Figuras/Correlaciones/Cuartetos}
	\caption{}
	\label{fig:cuartetos}
\end{figure}


\section{Intermitencia} 


\missingfigure{ESQUEMA INTERMITENCIA}



\begin{figure}[!ht]
	\centering
	\includegraphics[width=0.9897\linewidth]{Figuras/Intermitencia/PDFS/PDFS_DIFERENCIAS}
	\caption{}
	\label{fig:pdfsdiferencias}
\end{figure}


\begin{figure}[!ht]
	\centering
	\includegraphics[width=0.9897\linewidth]{"Figuras/Intermitencia/Funciones de Estructura/primeras_diferencias_v2"}
	\caption{}
	\label{fig:primerasdiferenciasv2}
\end{figure}


\begin{figure}[!ht]
	\centering
	\includegraphics[width=0.9897\linewidth]{"Figuras/Intermitencia/Funciones de Estructura/segundas_diferencias_v2"}
	\caption{}
	\label{fig:segundasdiferenciasv2}
\end{figure}

\missingfigure{CONVERGENCIA SP} 


\begin{figure}[!ht]
	\centering
	\includegraphics[width=0.9897\linewidth]{Figuras/Intermitencia/Exponentes/Exponentes_comparación_v2}
	\caption{}
	\label{fig:exponentescomparacionv2}
\end{figure}

 