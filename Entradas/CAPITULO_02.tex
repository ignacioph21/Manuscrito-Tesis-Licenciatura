\chapter{Sensor capacitivo} 




\begin{figure}[th!]
	\centering
	\includegraphics[width=0.987\linewidth]{Figuras/Sensor/Timeline/Timeline_v2}
	\caption{Diagrama esquemático de los pasos para la construcción y caracterización del sensor capacitivo para la altura de la superficie libre.}
	\label{fig:timelinev2}
\end{figure}


\clearpage 

\section{Principio de funcionamiento} 
En esta primera Sección se desarrollarán los fundamentos teóricos en los que se basa el funcionamiento del sensor capacitivo construído durante la Tesis, haciendo hincapie en las distintas hipótesis que deben cumplirse para que su respuesta sea lineal. En primer lugar se modela la sonda, componente  que irá sumergida y estará en contacto con el fluido, siguiendo por el circuito al cual se conecta, y terminando con la técnica de demodulación homodina, que una vez implementada permitirá realizar la medición propiamente dicha. % será la que  lugar  empezando por la rá 


\subsection{La sonda} 

Al sumergir un alambre de cobre esmaltado en un fluido conductor, como se esquematiza en la Figura \ref{fig:esquema_fundamento}, el recubrimiento aislante actuará a modo de medio dieléctrico entre ambos. En el caso de dos conductores cilíndricos coaxiales, de radios $r_1<r_2$, cuyo intersticio se encuentra lleno de un medio con permitividad $\varepsilon$ y permeabilidad $\mu$, se tienen una capacitancia ($C'$) e inductancia ($L'$) por unidad de longitud, dadas por % \cite{gordillozavaletaNonpropagatingHydrodynamicSolitons2012} % Para de  los conductores 

\begin{subequations}
	\begin{equation}
		C' = 2\pi\varepsilon\ln^{-1}\left(\frac{r_2}{r_1}\right)
	\end{equation} 
	\begin{equation}
		L' = \frac{\mu}{2\pi}\ln\left(\frac{r_2}{r_1}\right)
	\end{equation} 
\end{subequations}

\begin{figure}[th!]
	\centering
	\includegraphics[width=0.71089\linewidth]{"Figuras/Sensor/Esquema fundamento/esquema_fundamental"}
	\caption{Esquema de un alambre esmaltado sumergido en un fluido conductor, puesto a Tierra mediante otro alambre sin aislar. El alambre de cobre se encuentra a conectado a un potencial externo $V_{ext}$. El recubrimiento aislante actuará como medio dieléctrico, comportándose el sistema  efectivamente como un capacitor, cuya capacitancia depende de la altura sumergida. } 
	\label{fig:esquema_fundamento}
\end{figure}

  
La porción sumergida del alambre puede modelarse como una sucesión de segmentos de longitud $dl$, cada uno de los cuales agrega al circuito una capacitancia $dC = C'dl$ en paralelo y una inductancia $dL = L'dl$ en serie. De esta forma, es posible escribir la impedancia $Z(l+dl)$ recursivamente, a partir de la impedancia $Z(l)$ y la del segmento infinitesimal siguiente:  

\begin{equation}
	Z(l+dl) = j\omega L'dl + \left(\frac{1}{Z(l)} + \frac{1}{j\omega C' dl}\right)^{-1} 
\end{equation}  

Con $j$ la unidad imaginaria. Tomando $dl\rightarrow0$ y reteniendo términos hasta primer orden, resulta en una ecuación diferencial para $Z(l)$ 

\begin{equation}
	\frac{dZ}{dl} = j\omega L' - j\omega C'Z^2
\end{equation} 

Ésta puede resolverse agregando la condición de contorno $Z(0)=\infty$, que refleja la ausencia de un camino capacitivo hacia el fluido si el alambre no está sumergido, pudiendo así considerar al circuito como abierto. El resultado es el siguiente % sumando 

\begin{equation}
	Z(l) = j\sqrt{\frac{L'}{C'}} \tan\left(\omega \sqrt{L'C'} l - \pi/2\right)
\end{equation}

En el caso particular en que $l$ es lo suficientemente pequeño ($\omega\sqrt{\mu\varepsilon} l\ll 1$) resulta suficiente con tomar la expansión de esta solución a primer orden 

\begin{equation}
	Z(l) \approx \frac{1}{j\omega C' l} + \mathcal{O}\left(\frac{1}{L'C'^2\omega^2l^3}\right)
\end{equation}

En este régimen el sistema se comportará básicamente como un capacitor de capacitancia $C'l$, de modo que, midiendo la capacitancia de este circuito, será posible inferir la altura sumergida del alambre y, por tanto, la altura de la superficie libre. % De este modo, se puede observar que e 

Ahora bien, más que la longitud total sumergida del alambre, o sonda, la cantidad de interés será el apartamiento $\Delta l$ respecto de una posición de equilibrio $l_{0}$ (para la superficie libre en reposo), tal que $l=l_{0}+\Delta l$. Teniendo esta consideración en cuenta, la capacitancia de la sonda será $C_s = C_{0} + \Delta C$, y el objetivo de la medición será esta $\Delta C$, directamente relacionada a los desplazamientos de la superficie libre. % o 



Un último punto importante es que, si bien el desarrollo hasta aquí se llevó a cabo suponiendo casos donde el fluido alrededor del alambre se mantuviera estático, será posible levantar ésta hipótesis, y considerar que $\Delta l=\Delta l(t)$, siempre que las variaciones sean lo suficientemente lentas respecto al tiempo de respuesta típico del sistema: % la superficie libre a un  

\begin{equation}
	\tau_s = \alpha\frac{\varepsilon}{\sigma}\left(\frac{r_1}{r_2-r_1}\right) % +   
\end{equation}

Donde $\alpha$ es una constante que depende de la geometría de los conductores, y $\sigma$ es la conductividad del fluido. 

Se asumirá que esta condición se cumple en general. El único caso en que fue posible observar una limitación en el funcionamiento del sensor vinculada a ésto fue al aumentar varios órdenes de magnitud la conductividad del fluido  (agregando sal al agua destilada, en el orden de $10^{4}$ ppm) aunque para agua el agua de red el funcionamiento fue, hasta donde se pudo observar, equivalente al del agua destilada, a diferencia de lo reportado en \cite{gordillozavaletaNonpropagatingHydrodynamicSolitons2012}. % en  3-


% Esta condición implica que el sensor podría no funcionar correctamente en aguas 
% Se puede mostrar que, en general, esta condición se cumplirá para agua destilada \cite{gordillozavaletaNonpropagatingHydrodynamicSolitons2012}. % para e agua destilada éste $$ válida 





\subsection{Medición de la capacitancia} 
En la Sección anterior se mostró cómo la medición de la capacitancia de un alambre esmaltado sumergido permite determinar los desplazamientos de la superficie libre respecto de una posición en equilibrio.  % longitud que éste se encuentra inmerso ($l$) a 


\begin{figure}[th!]
	\centering
	\includegraphics[width=0.57018\linewidth]{"Figuras/Setups/circuito_completo v3"}
	\caption{Esquema del circuito utilizado para medir la diferencia de capacitancia $\Delta C$, y en consecuencia $\Delta l$, del sensor sumergido. Se conecta en paralelo la sonda al capacitor de un circuito RLC serie, de forma tal que el agua queda conectada a Tierra y el alambre de cobre a la otra punta de $C$. Se alimenta al circuito mediante una fuente de tensión alterna a la frecuencia de resonancia, y se mide el voltaje sobre la fuente y la resistencia.} 
	\label{fig:circuitocompleto-v3}
\end{figure}

Esta medición puede llevarse acabo conectando la sonda al circuito de la Figura \ref{fig:circuitocompleto-v3}, que se corresponde a su conexión en paralelo al capacitor de un circuito RLC serie (de parámetros base $R$, $L$ y $C$), teniendo como efecto la modificación de la capacitancia efectiva del mismo. De esta forma se tendrá un circuito RLC serie equivalente, cuya capacitancia será ahora $C_{\text{EQ}}(t)=C+C_s(t)$. % _0   

% . % sumergido conectar la sonda 
% Este arreglo á lo cual rá el r  

Al conectar éste circuito a una fuente de tensión alterna de frecuencia angular $\omega$ y amplitud $V$ la ecuación para la corriente $I(t)$ vendrá dada por % corriente 

\begin{equation}
	V \cos(\omega t) = \frac{d^2I(t)}{dt^2} + \frac{R}{L} \frac{dI(t)}{dt} + \frac{1}{C_{\text{EQ}}(t)L} I(t)  
	\label{eq:circuit_eq}
\end{equation}

Donde $C_{\text{EQ}}(t) = C + C_0 + \Delta C(t)\equiv C_q + \Delta C(t)$, definiendo $C_q$ como la capacitancia efectiva cuando la superficie libre se encuentra en reposo. Si las variaciones en la capacitancia son lo suficientemente lentas respecto a la frecuencia se puede asumir una solución de la forma $I(t)=I_0(t)\cos(\omega t + \phi(t))$, en analogía a lo que ocurriría en el caso estático para $C_{\text{EQ}}=\text{cte}$.  Cuando las variaciones de la capacitancia sean lo suficientemente pequeñas ($\Delta C/C_q\ll 1$) será suficiente con  un desarrollo a primer orden para describir la fase $\phi(t)$:      % pod puede escribirse _{\text{EQ}}  como 

\begin{equation}
		\tan(\phi) = -\frac{1}{R} \left[\omega L - \frac{1}{\omega C_q}\right] - \frac{1}{R} \frac{\Delta C}{\omega C_q^2} + \mathcal{O}\left[\left(\frac{\Delta C}{C_q}\right)^2\right] 
\end{equation}

Es posible eliminar además el primer término de este desarrollo si se elige trabajar con la frecuencia de resonancia, $\omega_r=1/\sqrt{LC_q}$. De esta forma quedaría simplemente 

\begin{equation}
	\tan(\phi) = -Q_F \frac{\Delta C}{C_q} 
\end{equation}

Donde $Q_F=\frac{1}{R}\sqrt{L/C_q}$ es el factor de calidad del RLC. Finalmente, si el término de la derecha también es mucho menor a uno, se puede aproximar $\tan(\phi)\approx\phi$ y queda una relación lineal entre $\phi$ y la variación de capacitancia, y por tanto, con las variaciones en la altura sumergida del sensor. 

\begin{equation}
	\phi(t) = -Q_{F} \frac{\Delta C(t)}{C_q} = -Q_{F}\frac{C'}{C_q} \Delta l(t) 
	\label{eq:relación_lineal}
\end{equation} 

Al ajustar los parámetros del RLC (y $Q_F$ en consecuencia) es posible ajustar la sensitividad del sensor, ya que si, por ejemplo, se quiere medir un rango muy amplio de $\Delta l$, bastaría con hacer que $Q_{F}$ sea más pequeño. % p  Y a 







\subsection{Medición de la fase} 
Hasta aquí se demostró conceptualmente cómo es posible encontrar una relación lineal entre la fase ($\phi$) de la corriente de la sonda y las variaciones en su altura sumergida ($\Delta l$) al conectarla en paralelo al capacitor de  un RLC serie. El paso final para completar la descripción del sensor será, entonces, incorporar un método para la medición de dicha fase. % do m o desplazamient   \textbf{ hallar} ty. 

Con este fin se empleará la técnica de \textit{demodulación homodina}, que utiliza una señal de referencia para obtener la amplitud y fase relativa de la señal de interés (en este caso la caída de tensión sobre la resistencia $R$). % ó 

A continuación se describe brevemente la técnica, mostrando en la Figura \ref{fig:ejemplo-v2} las señales involucradas en el proceso para un caso particular a modo ilustrativo. El primer paso consistirá en modular la señal de interés, a la cual se llamará $A(t)$, con alguna frecuencia, obteniendo entonces una nueva señal % a y 

\begin{equation}
	S(t) = A(t)\cos(\omega_r t)
\end{equation} 

Es importante que la frecuencia de modulación sea mucho más rápida que las frecuencias características de $A(t)$, para que pueda captar todas sus variaciones correctamente. En este caso, la tensión sobre la resistencia ya tendrá la misma forma de $S(t)$, con lo cual no es necesario realizar este primer paso explícitamente. % e  

Luego, se debe \textit{mixear} la señal  modulada con la de referencia, $R(t)=\cos(\omega_r t)$, que en este caso será la tensión alterna que alimenta al circuito\footnote{\normalsize La referencia podría tener en principio una frecuencia distinta a la de modulación, pero por simplicidad, y porque en este sistema será así, se considerarán iguales.},  obteniendo entonces 

\begin{equation}
	M(t) = S(t)R(t) = A(t)/2 \; \big(1+\cos(2\omega_r t)\big) 	
\end{equation} % T 

Si hubiese ruido eléctrico $N(t)$ sumando, la señal sería $M(t) = S(t)R(t) + N(t)R(t)$. El efecto de este mixeo será centrar el ruido en $\omega_r$, recordando que un producto en tiempo se traduce como una convolución en frecuencia. Además, quedan dos copias de la señal de interés, una centrada en 0 y otra en $2\omega_r$, como se puede ver en la Figura \ref{fig:ejemplo-v2}e. 

Finalmente, se debe aplicar un filtro pasabajos, de frecuencia de corte $\omega_c$, o tiempo de integración $\tau_c=2\pi/\omega_c$. Siempre que $\omega_c<\omega_r$ el efecto de este filtro será no dejar pasar, ni la copia de la señal en $2\omega_r$, ni al ruido, y obtener así la amplitud $A(t)/2$.   % por además total y , 

% \missingfigure{Esquema Lockin } 

\begin{figure}[th!]
	\centering
	\includegraphics[width=0.987\linewidth]{"Figuras/Sensor/lockin fase/Ejemplo v2"} % 1 
	\caption{Distintas señales asociadas a la técnica de demodulación homodina. En primer lugar (a) la señal a medir $A(t)$ y la señal de referencia $R(t)$. Con éstas se construyen (b) la señal modulada $S(t)$ y (c) la señal mixeada $M(t)$, a partir de la cual se obtiene (d) la reconstrucción de la señal de interés. Abajo de todo (e) se muestra el contenido en frecuencia de $M(t)$, pudiéndose apreciar a frecuencias bajas el contenido de $A(t)$ y otras dos copias del mismo alrededor de $2\omega_r$, que serán eliminadas mediante el filtro pasabajos, de frecuencia $\omega_c$, junto al ruido que pudiese haber.} 
	\label{fig:ejemplo-v2} % 1 
\end{figure}


Si inicialmente la señal hubiese tenido una fase respecto a la señal de referencia, o sea, $S(t)=A(t)\cos(\omega_r t + \phi_r)$, como ocurrirá en este caso, habría que repetir el mismo procedimiento con una señal de referencia que tenga 90$^\circ$ de desfasaje respecto a $R(t)$, a la cual se la conoce como \textit{cuadratura}. Al final se obtendrá para la referencia una señal $X(t) = A(t)/2 \; \cos(\phi_r)$, que se conoce como componente homodina, y para la cuadratura, una señal $Y(t)=A(t)/2 \; \sin(\phi_r)$, que es la componente heterodina. Con éstas resulta fácil obtener la amplitud y fase como: 

\begin{subequations}
	\begin{equation}
		A^2 = 4 (X^2+Y^2) 
	\end{equation} 
	\begin{equation}
		\phi_r = \arctan(Y/X)
	\end{equation}
\end{subequations}  % se  a ,  y $$. 

Por lo general, se utiliza un amplificador lock-in para llevar a cabo esta técnica de manera síncrona, pero se discutirá más sobre su implementación más adelante. % á



\section{Resolución numérica del circuito} 
Antes de iniciar con la construcción del sensor en el laboratorio, se empezó caracterizando numéricamente el comportamiento del circuito descripto en las Secciones anteriores, esto es, la resolución de la ecuación \eqref{eq:circuit_eq}. El objetivo de este paso preliminar fue, por una parte verificar que la solución hallada bajo las distintas aproximaciones funcionase correctamente, pero además, poder probar distintas combinaciones de parámetros y determinar qué efectos tendrían éstas sobre el funcionamiento del sensor. A continuación se mencionan brevemente algunas de las conclusiones más relevantes que se pudieron observar. % la ecuación ss resolviendo para  ver s ba 

\begin{figure}[th!]
	\centering 
	\includegraphics[width=0.87\linewidth]{"Figuras/Sensor/Pruebas numéricas/Diferencias"}
	\caption{(a) Comparación de la altura sumergida, $\Delta l$, con la reconstruida, obtenida mediante la resolución numérica de la ecuación \eqref{eq:circuit_eq} y la técnica de demodulación homodina para el cálculo de la fase. (b) Diferencia entre ambas señales, de dos órdenes de magnitud menor.} % &  ty  
	\label{fig:diferencias_num}
\end{figure}

Se pudo verificar que la fase se relaciona linealmente con $\Delta l$ hasta cierto punto. Por ejemplo, para una señal aleatoria (amplitud constante entre 0-6 Hz y fase aleatoria) como las que podrían ser de interés medir en la superficie libre, con los valores de la Tabla \ref{tab:rlc_valores} que se terminaron eligiendo para la construcción del sensor, la reconstrucción de la altura es la de la Figura \ref{fig:diferencias_num},  donde los errores son del orden del 1 \% aproximadamente.  % .Dss  de esta forma  eventualmente usando 

% \missingfigure{Saturación}
\begin{figure}[th!]
	\centering
	\includegraphics[width=0.57\linewidth]{"Figuras/Sensor/Pruebas numéricas/Variar Qs"}
	\caption{Comparación de las curvas de respuesta del sensor obtenidas numéricamente para distintos factores de calidad $Q_F$ (líneas sólidas). A menor factor de calidad se extiende el rango de alturas en el que funciona linealmente, manteniéndose cerca de la respuesta ideal (línea punteada). Factores de calidad mayores permiten medir diferencias de altura más pequeñas y saturan antes. En todos los casos para $\Delta l$ lo suficientemente pequeño el comportamiento es lineal  (recuadro ampliado). }
	\label{fig:variar-qs} % veunte 
\end{figure} 


Si los desplazamientos  $\Delta l$ se vuelven lo suficientemente grandes, como para que $\Delta C\sim C_q$, se produce una saturación en la fase medida.  Se puede contrarrestar este efecto como ya se explicó modificando los parámetros del circuito, siendo la opción más sencilla disminuir el factor de calidad $Q_F$, como se muestra en la Figura \ref{fig:variar-qs}, o bien disminuyendo la capacitancia por unidad de longitud $C'$ del alambre, lo cual implicaría cambiarlo por otro. % yendo  % se intentade a partir del cual a 










Otro efecto interesante que se pudo observar es que, si se intentan medir variaciones en $\Delta l$ demasiado rápidas respecto a la frecuencia de resonancia del circuito, se produce histéresis en el sistema. En la Figura \ref{fig:histeresis} se comparan dos ejemplos para la medición de una señal sinusoidal, una a frecuencia baja en comparación a la de resonancia (\ref{fig:histeresis}a) donde la respuesta es lineal y otra a una frecuencia más alta (\ref{fig:histeresis}b), donde se puede ver un comportamiento histerético en el sistema. % pa% ($f/f_r\sim10^{-4}$)  o 


% \missingfigure{Histéresis} 
\begin{figure}[th!]
	\centering
	\includegraphics[width=0.87\linewidth]{"Figuras/Sensor/Pruebas numéricas/Histéresis"}
	\caption{Arriba, la comparación de los desplazamientos $\Delta l$ reales y los obtenidos numéricamente, para señales senoidales de $f=0.2$ KHz (a.I) y $f=2$  kHz (b.I), con un circuito de frecuencia de resonancia de $f_s=67$ kH. Debajo, las curvas de respuesta (\marksolid[blue]{0.6pt}), an ambos casos comparadas a la ideal (\markdashdot[black]{0.6pt}), pudiendo observar una respuesta lineal para frecuencias relativamente bajas (a.II) en comparación a frecuencias más altas (b.II) donde el circuito no llega a responder lo suficientemente rápido y se produce histéresis. } % línea azul sólida a a  respuesta  $-$$\cdot-$ 
	\label{fig:histeresis}
\end{figure}


Aunque ninguno de estos efectos adversos se llegaron a observar en la práctica. 



\section{Prueba de un circuito equivalente} 
Siguiendo con el enfoque constructivo, el siguiente paso fue realizar una prueba conceptual del circuito. Para esto se armó un circuito RLC serie, a cuya capacitancia se le fue conectando en paralelo capacitancias menores de distintos valores, y midiendo en cada caso la diferencia de fase respecto de la señal de referencia. 

La medición de la diferencia de fase se llevó a cabo mediante un osciloscopio en este caso ya que todavía no se había puesto en práctica la técnica de demodulación homodina. 

\begin{figure}[th!]
	\centering
	\includegraphics[width=0.50787\linewidth]{"Figuras/Sensor/Prueba protoboard/Datos_osciloscopio_unidades"}
	\caption{Diferencia de fase $\Delta \phi$ obtenida al agregar en paralelo a la capacitancia de un RLC serie distintas capacitancias (\markcircle*[blue]{0.365ex}). La línea vertical (\marksolid[blue]{0.4ex}) separa los rangos donde el sistema se comporta o no linealmente. Para diferencias de fase pequeñas se muestra además el ajuste lineal (\marksolid[black]{0.4ex}) de los datos, con parámetros $a=(5.3\pm0.3)\times10^{-2}$ rad/pF y $b=(-5\pm6)\times10^{-3}$ rad. La extensión de dicho ajuste al otro rango (\markdashed{0.54ex}) muestra cómo se separan los datos significativamente.} % \textbf{ }{filled} \textbf{\bluecircle} mente
	\label{fig:datososciloscopiounidades}
\end{figure}


Para esta prueba se utilizaron los siguientes elementos, $R=(1800\pm200)\;\Omega$, $L=(10\pm1)$ mH y $C=(100\pm10)$ pF, que dan un factor de calidad $Q_F=(5.6\pm0.7)$. Éstos son similares a los que eventualmente se terminarían eligiendo para el sensor. 

En la Figura \ref{fig:datososciloscopiounidades} se pueden apreciar los resultados. Igual que lo que se había observado numéricamente, la respuesta del sistema es lineal para $Q_F\Delta C/C\ll1$, y empieza a desviarse cuando la capacitancia agregada a la del RLC serie se vuelve demasiado grande. % saturar  A

En el rango en que se observó un comportamiento lineal se ajustaron linealmente los datos, obteniendo una pendiente $a=(5.3\pm0.3)\times10^{-2}$ rad/pF, y una ordenada $b=(-5\pm6)\times10^{-3}$ rad. Como es de esperarse, la ordenada es compatible con cero, y la pendiente puede compararse  con la que se esperaría por la ecuación \eqref{eq:relación_lineal}, esto es, 

\begin{equation}
	a_{\text{teo}}=\frac{1}{R}\sqrt{\frac{L}{C^3}}
\end{equation}

Para los elementos elegidos resulta en $a_{\text{teo}}=(5.6\pm0.7)\times10^{-2}$ rad/pF, que coincide con el obtenido del ajuste lineal. Además los estimadores resultaron $R^2=0.99550$ y $\chi_\ni^2=1.25$, concluyendo que en este rango son bien descriptos los datos por una función lineal. 

De esta forma, se verifica el comportamiento predicho para el circuito en el caso estático. % la validez de la teoría 


\section{Diseño y prototipado del sensor} %  y prueba  P 
Ya habiendo terminado con la fase más exploratoria y las pruebas preliminares descriptas en las Secciones anteriores, se pasó al diseño y prototipado del sensor, al cual corresponde esta Sección, que implica el armado de la sonda y del circuito para la medición de la fase. La puesta en marcha de la técnica de demodulación homodina para llevar a cabo esta medición será descripta en la Sección siguiente debido a la complejidad que terminó requiriendo. % , junto a l s  ya que tuvo su 

\subsection*{La sonda} 
Un modelo de la versión final de la sonda se muestra en la Figura \ref{fig:dibujosensormodelo}. Ésta es relativamente sencilla, siendo su componente principal el alambre de cobre esmaltado, que en este caso fue de $(0.35\pm0.05)$ mm de diámetro. 

\begin{figure}[th!]
	\centering
	\includegraphics[width=0.87\linewidth]{"Figuras/Sensor/El sensor/dibujo_sensor_modelo"}
	\caption{}
	\label{fig:dibujosensormodelo}
\end{figure}

Las partes restantes son básicamente para el soporte estructural del alambre, y para facilitar su conexión al circuito RLC. 

El alambre de acero en forma de U, además de dar un lugar para el anclaje del alambre de cobre, tiene la funcionalidad de poner al fluido circundante a la misma Tierra que el circuito. 

La carcasa (en negro) fue diseñada como dos partes que se conectan entre sí mediante cuatro tornillos M4. Estas dos mitades se imprimieron mediante impresoras 3D en plástico PLA. Se eligió hacerla en dos mitades ya que en el centro se encuentran las canaletas por donde se enhebran ambos alambres, que de otra forma hubiesen sido difíciles de lograr con la impresora utilizada, y de esta forma además es más fácil su inserción previo a atornillarlas. % o  É 

La parte superior de la carcasa se encuentra abierta, exponiendo así ambos alambres y permitiendo su conexión mediante cables de terminaciones tipo cocodrilo a la placa con el circuito RLC serie. 

Por último, la cruz en el medio (\textit{trusses} en inglés) dan mayor integridad estructural a la parte inferior de la carcasa, cuyo fin es dar un segundo punto de anclaje al alambre de cobre y asegurarse así que se mantenga recto. Cualquier deformación en la forma de éste podría traducirse en una curva de calibración no lineal que complicaría las mediciones. Algo similar ocurriría si el alambre entra de forma no perpendicular en el fluido, pero esto último solo tiene como consecuencia una constante multiplicativa en la constante de proporcionalidad entre la fase y $\Delta l$ (relacionada a la proyección sobre la línea perpendicular), con lo cual la respuesta se mantendría lineal. % Lo mismo éste 

Un punto importante a tener en cuenta es que, al cortar el alambre de cobre para montar la sonda, la parte que quedará sumergida tiene en la punta al alambre expuesto sin esmaltar, y la corriente tomaría este camino hacia la Tierra en lugar del capacitivo a través del recubrimiento aislante. Para evitar esto se aplicó un spray aislante (Ailamatic DELTA), que además al secarse afianzó la adherencia al alambre de acero. 
 

\subsection*{El circuito RLC serie} 
Los valores nominales de los elementos utilizados para el sensor son los de la Tabla \ref{tab:rlc_valores}. Estos valores se eligieron luego de varias pruebas preliminares, ya que fueron los que mejores resultados permitieron obtener para los desplazamientos típicos de la superficie libre con los que se trabajó ($\lesssim$1 cm).   % de un orden máximo de  a 

\begin{table}[h]
	\centering
	\begin{tabular}{|c|c|}
		\hline
		\textbf{Elemento} & \textbf{Valor} \\ \hline
		R & $(360 \pm 40)\,\Omega$ \\ \hline
		L & $(10 \pm 1)\,\text{mH}$ \\ \hline
		C & $(560 \pm 60)\,\text{pF}$ \\ \hline
	\end{tabular}
	\caption{.}
	\label{tab:rlc_valores}
\end{table} 


Es importante resaltar que, como muestra la Figura \ref{fig:circuitocompleto-v3}, el elemento a Tierra en el RLC debe ser el capacitor, ya que el agua estará a este potencial en cualquier caso, y de conectar, por ejemplo, la resistencia a Tierra, se generaría un cortocircuito. Se observó que, al tener este elemento  a Tierra en el RLC (independientemente de si la sonda estaba conectada), se produce una disminución en la frecuencia de resonancia del mismo, que se atribuye al acoplamiento de capacitancias parásitas $C_p$, que tienen como efecto sumar a la $C$ utilizada ($C_p\;\sim$ 200 pF), con lo cual se debe asumir que es levemente mayor al valor reportado en la Tabla \ref{tab:rlc_valores}.  % de un orden similar   el capacitor


De esta manera, para el sensor en el aire la capacitancia total $C_{\text{EQ}}$ resultó ser aproximadamente 400 pF mayor a la $C$ base del RLC, con lo cual $C_{\text{EQ}}\;\sim$ 1000 pF. Con el sensor sumergido aproximadamente 1 cm en agua destilada la variación en la capacitancia es del orden de $\Delta C\;\sim$ 10 pF, de forma que se verifica la hipótesis que hicimos anteriormente para llegar a una relación lineal entre la fase y los desplazamientos de altura, resultando el cociente $\Delta C/C\sim0.01\ll1$.   % \approx 

La capacitancia por unidad de longitud para el cable de cobre elegido se estimó a partir de estos valores en el orden de $1$ pF/mm. % $0^{-12}$ 


Las frecuencias de resonancia típicas con las que se trabajó fueron del orden de $f_s\sim52$ kHz, para los elementos previamente mencionados, ajustándolas más finamente en cada caso, antes de la medición, aunque cuando por distintos motivos no fue posible realizar este ajuste el efecto fue simplemente la suma de un $\phi_0$ a las señales, que puede eliminarse al restar el valor medio, ya que finalmente la cantidad de interés será la variación de altura. % n a los fines . % En algunos casos se observó una deriva muy lenta  



% De esta manera, a capacitancia total $Cq$ para el sensor sin sumergir resultó de , con una frecuencia de resonancia de $f_s=(53.19\pm0.01)$ kHz, mientras que con el sensor sumergido fue de $C_q=(895\pm90)$ pF y una frecuencia $f_s=(52.60\pm0.01)$ kHz. 

%Para el sensor en el aire la capacitancia total $C_q$ que se debe sumar al $C$ base resultó ser del orden de 400 pF, a la cual contribuye entonces la $C_0$ y las capacitancias parásitas, que se estiman del orden de 200 pF. De esta forma la capacitancia total Al sumergir aproximadamente 1 cm el sensor en agua destilada se observa una variación en la capacitancia de entre 10 y 20 pF, lo cual sería consistente con la hipótesis de $C/\Delta C\ll 1$. 
%
%Las frecuencias de resonancia típicas con las que se trabajó fueron en consecuencia del orden de 52 kHz. 



\section{Placa de adquisición y Lock-in digital} 
Las primeras pruebas de funcionamiento del sensor se llevaron a cabo utilizando un amplificador lock-in SR-830, lo cual permitió verificar que el comportamiento era el esperado. Sin embargo, se abandonó el uso de este lock-in antes de seguir con la caracterización, ya que éste presentaba dos factores que complicarían su aplicación en el largo plazo. En primer lugar son equipos grandes de difícil movilidad, pero lo principal es que cuentan con un solo canal de medición, con lo cual no es escalable para múltiples sensores, que sería la idea poder usar en el laboratorio eventualmente. 

% Por un lado, los dos amplificadores disponibles para préstamo en los laboratorios de enseñanza no pueden ser usados a voluntad, ya que depende de los horarios de clases en cada momento. 

Se exploró la idea de utilizar distintas placas de desarrollo programables que aplicasen el algoritmo de demodulación homodina de forma digital y sincrónica, por ejemplo Arduino UNO, pero su limitación principal fue la "lenta" tasa de adquisición, respecto a las frecuencias de oscilación de las tensiones a medir.   

Finalmente se optó por utilizar una placa de adquisición para medir simultáneamente las tensiones a una alta tasa y posteriormente analizar las señales digitalmente (ya sea de forma síncrona o asíncrona luego de la medición). 



\subsection*{La placa de adquisición}

La placa utilizada fue la disponible de mayor tasa en el conversor analógico digital, que fue la PersonalDAQ 3001 de IOTech \cite{MCC_PersonalDAQ3000_Manual}. Ésta placa permite adquirir con una tasa de hasta 1 MHz distribuida en 8 canales en formato diferencial (o 16 canales puntuales) cada uno con una resolución de 16 bits para un rango de voltaje ajustable entre $\pm10$ V y $\pm 100$ mV. % 0 

Para las mediciones de esta Tesis se utilizó siempre un único sensor, con lo cual fueron necesarios solo dos canales en modo diferencial, uno para la referencia y otro para la tensión sobre la resistencia, de modo que cada canal tuvo una frecuencia de muestreo máxima de 500 kHz. 

Si bien puede parecer mucho una adquisición a esta tasa para finalmente medir oscilaciones en la superficie libre de hasta unos 200 Hz hay que recordar que éstas estarán moduladas por la frecuencia de resonancia del RLC serie, que como se mencionó en la Sección anterior es del orden de 50 kHz, con lo cual la frecuencia mínima de muestreo (por el Teorema de Nyquist para evitar aliasing) está en 100 kHz. Los 500 kHz permiten estar por encima de este mínimo por una cantidad considerable, pero pruebas con tres sensores en simultáneo, que requiere un total de cuatro canales en modo diferencial, a 250 kHz cada uno dieron resultados comparables a los de una tasa mayor.  


\subsection*{Software de control de la placa}
La principal complejidad que vino asociada a la utilización de esta placa fue su comunicación con la computadora para llevar a cabo las mediciones, ya que la compañía no proporciona software oficial compatible con sistemas operativos modernos (la última versión disponible es para Windows XP). 


% Fue necesario entonces desarrollar las rutinas de adquisición síncrona y asíncrona para la placa, entre otras, prácticamente desde cero, a partir de los comandos básicos que se encuentran en los controladores de la placa. 

Fue necesario entonces desarrollar rutinas propias para la adquisición síncrona y asíncrona, entre otras, a partir de las funciones de bajo nivel expuestas por la biblioteca dinámica, que se comunica con el controlador del dispositivo. % actúa como interfaz  

Con  este fin se utilizó Python 3, y se basó el desarrollo en el wrapper de la biblioteca dinámica \textit{PyIOTech} \cite{PyIOTech}, que también fue necesario readaptar ya que se encuentra escrita en Python 2. 

No se entrara en detalle sobre el código en sí, pero el resultado final fue una interfaz gráfica de usuario mínima para llevar a cabo las mediciones en los canales deseados, que se encuentra disponible en \cite{Hernando_PersonalDAQ3001_Python}. Se implementaron dos métodos de adquisición, en primer lugar uno para una cantidad de puntos predeterminados, ideal para mediciones cortas, y uno pensado para mediciones largas que cuenta con guardado en disco en simultáneo a la medición por bloques, más eficiente que el primero al utilizar un buffer circular pequeño y no necesitar guardar todos los puntos en memoria antes de escribirlos en disco. Éste segundo método fue el más utilizado ya que permite tomar datos continuamente y detenerlo cuando se considere suficiente. 





\subsection*{El algoritmo de Lock-in Digital} 
Con el software desarrollado para la placa es posible entonces medir simultáneamente la señal de referencia y el voltaje sobre la resistencia del RLC serie. El análisis de estas señales se lleva a cabo posterior a su adquisición mediante los pasos explicados para la técnica de demodulación homodina, llegando efectivamente a lo que sería un lock-in digital asíncrono. Se desarrolló también una versión que funciona en tiempo real para el segundo método de adquisición explicado, pero que permite visualizar señales únicamente hasta unas pocas decenas de Hz. % aunque

Tampoco se entrará en detalle sobre la implementación de este algoritmo, pero se mencionarán brevemente algunos de los puntos principales que difieren al método convencional. Esto se debe a que por eficiencia en la adquisición solo se midió la referencia, y la cuadratura se generó digitalmente a partir de ésta.  

Para esto se implementó la Transformada de Hilbert, que tendrá el efecto de aplicar un desfasaje de 90$^\circ$ a la señal de referencia.  %  que para una señal limitada en frecuencia (a) 

Otro punto importante es que, al trabajar con señales de frecuencias tan altas no será despreciable el delay temporal que hay entre la medición de un canal y el siguiente de la placa, que en este caso se corresponde a $\Delta t=1$ $\mu$s. Este efecto es como si la referencia tuviese una fase adicional $\phi_0=\omega_s\Delta_t$, sobre la cual estarán montadas las variaciones que se quieren medir. Hay varias opciones para corregir esto, una es simplemente restar esto al final, pero no tiene en cuenta la posibilidad de pequeñas variaciones locales en la frecuencia o fase de la señal de referencia que se pueden tr a la fase final obtenida. Para tener esto en cuenta lo que se terminó optando por hacer es aplicar un \textit{fractional delay filter} (FDF) teniendo en cuenta que la señal es limitada en frecuencia. Esto permite reevaluar a la señal de referencia en los mismos tiempos que la señal del segundo canal, siendo fiel a la señal real a diferencia de lo que se puede obtener mediante una interpolación lineal. 

El último punto a mencionar tiene que ver con el filtro pasabajos utilizado. Se eligió una filtro de tipo exponencial, en base a lo explicado en \cite{harvieOLIAOpensourceDigital2023}, ya que permite extender el funcionamiento del Lock-in a tiempo real. Este filtro, que transcurre en dos etapas, usando el parámetro $\alpha=\cos(\gamma)-1+\sqrt{\cos^2(\gamma)-4\cos(\gamma)+3}$ que se relaciona con la frecuencia de corte o el tiempo de integración $\tau$ de la forma: $\gamma=2\pi f_c/f_s$ con $f_s$ la frecuencia de sampleo. Entonces:

\begin{align} 
	X_1[n] &= X_1[n-1] + \alpha(X_0[n]-X_1[n-1]) \\
	Y_1[n] &= Y_1[n-1] + \alpha(Y_0[n]-Y_1[n-1]) \\
\end{align}

y la segunda etapa: 

\begin{align}
	X_2[n] &= X_2[n-1] + \alpha(X_1[n]-X_2[n-1]) \\
	Y_2[n] &= Y_2[n-1] + \alpha(Y_1[n]-Y_2[n-1]) \\
\end{align}

Y podemos elegir trivialmente $X_0[-1]=Y_0[-1]=0$. Con este para la señal anterior y el FDF tenemos: 



El código para este lock-in digital se encuentra disponible en \cite{Hernando_DigitalLockIn}. La versión final permitió obtener resultados comparables a los de un Lock-in como el SR-830, con todas las ventajas de portabilidad y escalado que vienen asociadas a la placa. Un ejemplo clásico que muestra el correcto funcionamiento de éste es la medida de la amplitud y fase de un RLC serie en función de la frecuencia, como el que se presenta en el Apéndice \ref{apéndice:prueba_lockin_digital}.   % muestra : \missingfigure{Código lock-in digital} 



\section{Prueba final y calibración} 
Teniendo 

La frecuencia máxima 


\subsection{Montaje} 

\begin{figure}[th!]
	\centering
	\includegraphics[width=0.87\linewidth]{Figuras/Setups/Montaje_calibracion_fondo}
	\caption{}
	\label{fig:montajecalibracion}
\end{figure}

\subsection{Altura vs Altura} 

\begin{figure}[th!]
	\centering
	\includegraphics[width=0.87\linewidth]{"Figuras/Sensor/Series temporales/Series temporales a"}
	\caption{}
	\label{fig:series-temporales-a}
\end{figure}



\begin{figure}[th!]
	\centering
	\includegraphics[width=0.87\linewidth]{"Figuras/Sensor/Motor vs Fase/Motor vs FASE"}
	\caption{}
	\label{fig:motor-vs-fase}
\end{figure}


\subsection{Diferencias} 


\begin{figure}[th!]
	\centering
	\includegraphics[width=0.87\linewidth]{Figuras/Sensor/Diferencias/Diferencias}
	\caption{}
	\label{fig:diferencias}
\end{figure}


